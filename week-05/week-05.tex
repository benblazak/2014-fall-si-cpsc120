% vim: foldmethod=marker foldlevel=0

% -----------------------------------------------------------------------------
% unusual(?) dependencies
% -----------------------------------------------------------------------------
%
% - minted: version 2+
% - Pygments: for the minted package


% -----------------------------------------------------------------------------
% information
% -----------------------------------------------------------------------------

\def \docAuthor {Ben Blazak}
\def \docClass  {CPSC 120 SI}
\def \docSchool {California State University Fullerton}
\def \docTerm   {Fall 2014}
\def \docTitle  {Week 5 Worksheet}


% -----------------------------------------------------------------------------
% document setup
% -----------------------------------------------------------------------------
% {{{

\documentclass[12pt,letterpaper]{article}

\usepackage[includehead,
            includefoot,
            margin=1in,
            top=.25in,
            headheight=.75in,
            headsep=.25in,
            footskip=.25in,
           ]{geometry}

\usepackage[fleqn]{amsmath}
\usepackage{amssymb}
\usepackage{array}
\usepackage{enumitem}
\usepackage{fancybox}
\usepackage{fancyhdr}
\usepackage{l3regex}
\usepackage{mathtools}
\usepackage{minted}
\usepackage{multicol}
\usepackage{multirow}
\usepackage{tikz}
\usepackage[normalem]{ulem}
\usepackage{url}
\usepackage{xcolor}


% text ------------------------------------------------------------------------

\binoppenalty = 10000  % never break next to a binary operator
\relpenalty   = 10000  % never break next to a relation operator

\setlength{\parindent}{0em}
\setlength{\parskip}{1ex}

\setlist[itemize]{nosep,itemsep=.5ex,parsep=.5ex}

% math ------------------------------------------------------------------------

\setlength{\mathindent}{1cm}

% - "\begin{document}" resets these values, so they have to be treated
%   specially
\AtBeginDocument{
  \setlength{\abovedisplayskip}{1.5ex plus .5ex minus .5ex}
  \setlength{\belowdisplayskip}{1.5ex plus .5ex minus .5ex}
}

% source code -----------------------------------------------------------------

\usemintedstyle{solarizedlight}

% header and footer -----------------------------------------------------------

\pagestyle{fancy}

\lhead{\docClass}
\rhead{\docTitle}
\cfoot{\thepage}
\renewcommand{\headrule}{\hrule height 0.4pt}
\renewcommand{\footrule}{\hrule height 0.4pt}

\fancypagestyle{firstpage}{
  \fancyhead[L]{\docAuthor\\\docClass}
  \fancyhead[C]{\docSchool\\}
  \fancyhead[R]{\docTerm\\\docTitle}
}


% -----------------------------------------------------------------------------
% macros
% -----------------------------------------------------------------------------

% abbreviations ---------------------------------------------------------------

\def \<{\langle}
\def \>{\rangle}

\def \ε{\varepisilon}
\def \θ{\vartheta}
\def \κ{\varkappa}
\def \π{\varpi}
\def \ρ{\varrho}
\def \σ{\varsigma}
\def \φ{\varphi}

\def \Γ{\varGamma}
\def \Δ{\varDelta}
\def \Θ{\varTheta}
\def \Λ{\varLambda}
\def \Ξ{\varXi}
\def \Π{\varPi}
\def \Σ{\varSigma}
\def \Υ{\varUpsilon}
\def \Φ{\varPhi}
\def \Ψ{\varPsi}
\def \Ω{\varOmega}

% special characters ----------------------------------------------------------

\catcode `α = \active \let α \alpha
\catcode `β = \active \let β \beta
\catcode `γ = \active \let γ \gamma
\catcode `δ = \active \let δ \delta
\catcode `ε = \active \let ε \epsilon
\catcode `ζ = \active \let ζ \zeta
\catcode `η = \active \let η \eta
\catcode `θ = \active \let θ \theta
\catcode `ι = \active \let ι \iota
\catcode `κ = \active \let κ \kappa
\catcode `λ = \active \let λ \lambda
\catcode `μ = \active \let μ \mu
\catcode `ν = \active \let ν \nu
\catcode `ξ = \active \let ξ \xi
\catcode `ο = \active \let ο o
\catcode `π = \active \let π \pi
\catcode `ρ = \active \let ρ \rho
\catcode `σ = \active \let σ \sigma
\catcode `τ = \active \let τ \tau
\catcode `υ = \active \let υ \upsilon
\catcode `φ = \active \let φ \phi
\catcode `χ = \active \let χ \chi
\catcode `ψ = \active \let ψ \psi
\catcode `ω = \active \let ω \omega

\catcode `Α = \active \let Α A
\catcode `Β = \active \let Β B
\catcode `Γ = \active \let Γ \Gamma
\catcode `Δ = \active \let Δ \Delta
\catcode `Ε = \active \let Ε E
\catcode `Ζ = \active \let Ζ Z
\catcode `Η = \active \let Η H
\catcode `Θ = \active \let Θ \Theta
\catcode `Ι = \active \let Ι I
\catcode `Κ = \active \let Κ K
\catcode `Λ = \active \let Λ \Lambda
\catcode `Μ = \active \let Μ M
\catcode `Ν = \active \let Ν N
\catcode `Ξ = \active \let Ξ \Xi
\catcode `Ο = \active \let Ο O
\catcode `Π = \active \let Π \Pi
\catcode `Ρ = \active \let Ρ P
\catcode `Σ = \active \let Σ \Sigma
\catcode `Τ = \active \let Τ T
\catcode `Υ = \active \let Υ \Upsilon
\catcode `Φ = \active \let Φ \Phi
\catcode `Χ = \active \let Χ X
\catcode `Ψ = \active \let Ψ \Psi
\catcode `Ω = \active \let Ω \Omega

% functions -------------------------------------------------------------------

\def \ceil #1{\left\lceil#1\right\rceil}
\def \floor #1{\left\lfloor#1\right\rfloor}

% }}}
% other -----------------------------------------------------------------------
% {{{

% - sometimes a "\par", especially at the end of a block, is necessary to
%   prevent an extra (empty) paragraph from appearing in the output
% - `\color{.!50}` means 50 percent of the current color

     \def \note     #1{{\color{.!50}#1\par}}
\long\def \longnote #1{{\color{.!50}#1\par}}

\long\def \subquestion #1{
  \par #1 \par
}
\long\def \subproof #1{
  \vspace{1ex}\par\textbf{\textit{Proof.}} #1 \unskip\hfill$\square$\par\vspace{1ex}
}
\long\def \subsolution #1{
  \vspace{1ex}\par\textbf{\textit{Solution.}} #1 \unskip\hfill$\square$\par\vspace{1ex}
}

\long\def \question #1{\filbreak\subquestion{#1}}
\long\def \proof    #1{\subproof{#1}}
\long\def \solution #1{\subsolution{#1}}

% }}}
% -----------------------------------------------------------------------------
% document
% -----------------------------------------------------------------------------

\begin{document}
\thispagestyle{firstpage}

The website for these SI sessions is
\url{https://github.com/benblazak/2014-fall-si-cpsc120}.

Many of these examples are from
\url{https://github.com/benblazak/2014-spring-si-cpsc120},
which is full of stuff I wrote for a lab last semester.  If you're looking for
extra practice, this is one of many places you might start.


\filbreak
\section*{}
% TODO


\end{document}

