% vim: foldmethod=marker foldlevel=0

% -----------------------------------------------------------------------------
% unusual(?) dependencies
% -----------------------------------------------------------------------------
%
% - minted: version 2+
% - Pygments: for the minted package


% -----------------------------------------------------------------------------
% information
% -----------------------------------------------------------------------------

\def \docAuthor {Ben Blazak}
\def \docClass  {CPSC 120 SI}
\def \docSchool {California State University Fullerton}
\def \docTerm   {Fall 2014}
\def \docTitle  {Week 3 Worksheet}


% -----------------------------------------------------------------------------
% document setup
% -----------------------------------------------------------------------------
% {{{

\documentclass[12pt,letterpaper]{article}

\usepackage[includehead,
            includefoot,
            margin=1in,
            top=.25in,
            headheight=.75in,
            headsep=.25in,
            footskip=.25in,
           ]{geometry}

\usepackage[fleqn]{amsmath}
\usepackage{amssymb}
\usepackage{array}
\usepackage{enumitem}
\usepackage{fancybox}
\usepackage{fancyhdr}
\usepackage{l3regex}
\usepackage{mathtools}
\usepackage{minted}
\usepackage{multicol}
\usepackage{multirow}
\usepackage{tikz}
\usepackage[normalem]{ulem}
\usepackage{url}
\usepackage{xcolor}


% text ------------------------------------------------------------------------

\binoppenalty = 10000  % never break next to a binary operator
\relpenalty   = 10000  % never break next to a relation operator

\setlength{\parindent}{0em}
\setlength{\parskip}{1ex}

\setlist[itemize]{nosep,itemsep=.5ex,parsep=.5ex}

% math ------------------------------------------------------------------------

\setlength{\mathindent}{1cm}

% - "\begin{document}" resets these values, so they have to be treated
%   specially
\AtBeginDocument{
  \setlength{\abovedisplayskip}{1.5ex plus .5ex minus .5ex}
  \setlength{\belowdisplayskip}{1.5ex plus .5ex minus .5ex}
}

% source code -----------------------------------------------------------------

\usemintedstyle{solarizedlight}

% header and footer -----------------------------------------------------------

\pagestyle{fancy}

\lhead{\docClass}
\rhead{\docTitle}
\cfoot{\thepage}
\renewcommand{\headrule}{\hrule height 0.4pt}
\renewcommand{\footrule}{\hrule height 0.4pt}

\fancypagestyle{firstpage}{
  \fancyhead[L]{\docAuthor\\\docClass}
  \fancyhead[C]{\docSchool\\}
  \fancyhead[R]{\docTerm\\\docTitle}
}


% -----------------------------------------------------------------------------
% macros
% -----------------------------------------------------------------------------

% abbreviations ---------------------------------------------------------------

\def \<{\langle}
\def \>{\rangle}

\def \ε{\varepisilon}
\def \θ{\vartheta}
\def \κ{\varkappa}
\def \π{\varpi}
\def \ρ{\varrho}
\def \σ{\varsigma}
\def \φ{\varphi}

\def \Γ{\varGamma}
\def \Δ{\varDelta}
\def \Θ{\varTheta}
\def \Λ{\varLambda}
\def \Ξ{\varXi}
\def \Π{\varPi}
\def \Σ{\varSigma}
\def \Υ{\varUpsilon}
\def \Φ{\varPhi}
\def \Ψ{\varPsi}
\def \Ω{\varOmega}

% special characters ----------------------------------------------------------

\catcode `α = \active \let α \alpha
\catcode `β = \active \let β \beta
\catcode `γ = \active \let γ \gamma
\catcode `δ = \active \let δ \delta
\catcode `ε = \active \let ε \epsilon
\catcode `ζ = \active \let ζ \zeta
\catcode `η = \active \let η \eta
\catcode `θ = \active \let θ \theta
\catcode `ι = \active \let ι \iota
\catcode `κ = \active \let κ \kappa
\catcode `λ = \active \let λ \lambda
\catcode `μ = \active \let μ \mu
\catcode `ν = \active \let ν \nu
\catcode `ξ = \active \let ξ \xi
\catcode `ο = \active \let ο o
\catcode `π = \active \let π \pi
\catcode `ρ = \active \let ρ \rho
\catcode `σ = \active \let σ \sigma
\catcode `τ = \active \let τ \tau
\catcode `υ = \active \let υ \upsilon
\catcode `φ = \active \let φ \phi
\catcode `χ = \active \let χ \chi
\catcode `ψ = \active \let ψ \psi
\catcode `ω = \active \let ω \omega

\catcode `Α = \active \let Α A
\catcode `Β = \active \let Β B
\catcode `Γ = \active \let Γ \Gamma
\catcode `Δ = \active \let Δ \Delta
\catcode `Ε = \active \let Ε E
\catcode `Ζ = \active \let Ζ Z
\catcode `Η = \active \let Η H
\catcode `Θ = \active \let Θ \Theta
\catcode `Ι = \active \let Ι I
\catcode `Κ = \active \let Κ K
\catcode `Λ = \active \let Λ \Lambda
\catcode `Μ = \active \let Μ M
\catcode `Ν = \active \let Ν N
\catcode `Ξ = \active \let Ξ \Xi
\catcode `Ο = \active \let Ο O
\catcode `Π = \active \let Π \Pi
\catcode `Ρ = \active \let Ρ P
\catcode `Σ = \active \let Σ \Sigma
\catcode `Τ = \active \let Τ T
\catcode `Υ = \active \let Υ \Upsilon
\catcode `Φ = \active \let Φ \Phi
\catcode `Χ = \active \let Χ X
\catcode `Ψ = \active \let Ψ \Psi
\catcode `Ω = \active \let Ω \Omega

% functions -------------------------------------------------------------------

\def \ceil #1{\left\lceil#1\right\rceil}
\def \floor #1{\left\lfloor#1\right\rfloor}

% }}}
% other -----------------------------------------------------------------------
% {{{

% - sometimes a "\par", especially at the end of a block, is necessary to
%   prevent an extra (empty) paragraph from appearing in the output
% - `\color{.!50}` means 50 percent of the current color

     \def \note     #1{{\color{.!50}#1\par}}
\long\def \longnote #1{{\color{.!50}#1\par}}

\long\def \subquestion #1{
  \par #1 \par
}
\long\def \subproof #1{
  \vspace{1ex}\par\textbf{\textit{Proof.}} #1 \unskip\hfill$\square$\par\vspace{1ex}
}
\long\def \subsolution #1{
  \vspace{1ex}\par\textbf{\textit{Solution.}} #1 \unskip\hfill$\square$\par\vspace{1ex}
}

\long\def \question #1{\filbreak\subquestion{#1}}
\long\def \proof    #1{\subproof{#1}}
\long\def \solution #1{\subsolution{#1}}

% }}}
% -----------------------------------------------------------------------------
% document
% -----------------------------------------------------------------------------

\begin{document}
\thispagestyle{firstpage}

The website for these SI sessions is
\url{https://github.com/benblazak/2014-fall-si-cpsc120}.

Many of these examples are from
\url{https://github.com/benblazak/2014-spring-si-cpsc120},
which is full of stuff I wrote for a lab last semester.  If you're looking for
extra practice, this is one of many places you might start.


\filbreak
\section*{Some Code}
\begin{enumerate}

  \filbreak \item
    \inputminted{cpp}{hello-world-01.cpp}


  \filbreak \item
    This version has no \mintinline{cpp}{using namespace std;}, so
    \mintinline{cpp}{cout} must be fully qualified as
    \mintinline{cpp}{std::cout}.

    \inputminted{cpp}{hello-world-02.cpp}

    In the classroom, \mintinline{cpp}{using namespace std;} is almost always
    used.  In the real world, I've read that it's considered bad practice.  You
    should be aware that there are different ways to do things.  If the
    instructor cares, of course, you kind of have to write it how they wish.
    Past that though, just think about who you're writing for -- whether it's
    your friend (because what friend doesn't want to see your C++ projects?
    duh), or your future self in a few months when you're studying for your
    midterm.  Think about what they would find prettier and easier to read, and
    write it that way.


%   \filbreak \item
%     This version uses macros to define the message we print to stdout and the
%     value (the integer, or \mintinline{cpp}{int}) we return to the operating
%     system.
% 
%     \inputminted{cpp}{hello-world-03.cpp}


%   \filbreak \item
%     Here, we only return a value, we don't \mintinline{cpp}{cout} anything.  As
%     a consequence we do not need the \mintinline{cpp}{#include <iostream>} at
%     the beginning.  How do you think this would effect the preprocessed and
%     compiled files?
% 
%     \inputminted{cpp}{hello-world-04.cpp}


  \filbreak \item
    Here, we only return a value, we don't \mintinline{cpp}{cout} anything.  As
    a consequence we do not need the \mintinline{cpp}{#include <iostream>} at
    the beginning.  How do you think this would effect the preprocessed and
    compiled files?

    What if we return a value other than 0?

    \inputminted{cpp}{hello-world-05.cpp}

    In real code, returning anything other than \mintinline{cpp}{0} from
    \mintinline{cpp}{main()} usually means that an error occurred.

    Notice that \mintinline{cpp}{42} and \mintinline{cpp}{0} are both integers.
    The \mintinline{cpp}{int} in front of \mintinline{cpp}{main()} is what
    tells the compiler that we're planning to return an integer.  Since
    \mintinline{cpp}{main()} is a special function -- it's the function that
    the operating system calls when you ask it to run your program -- it must
    have this signature (i.e.~it must return an integer).  Other functions can
    return different things.


%   \filbreak \item
%     This version has a \mintinline{cpp}{return} before our
%     \mintinline{cpp}{cout}.  How does that change things?
% 
%     \inputminted{cpp}{hello-world-06.cpp}


  \filbreak \item
    Wait, we can have more than one \mintinline{cpp}{return}?

    \inputminted{cpp}{hello-world-07.cpp}

    Having more than one \mintinline{cpp}{return} in a function won't be useful
    until we talk about control flow
    (e.g.~\mintinline{cpp}{if}\ldots\mintinline{cpp}{else} statements), and
    some people consider it dangerous.  In general, I think it's okay, as long
    as it makes your code more clear -- which is to say, as long as you're
    \textbf{careful}; as with everything :) .


%   \filbreak \item
%     Variables :) .  Like in math, variables are names that can be assigned
%     different values.  Unlike in math, variables can have different values at
%     different points in time.
% 
%     What value is returned?
% 
%     \inputminted{cpp}{hello-world-08.cpp}


%   \filbreak \item
%     Another example about variables.  More on variables after we've talked
%     about them more in class.
%     
%     \inputminted{cpp}{hello-world-09.cpp}

\end{enumerate}


\filbreak
\section*{Variables, Basic Operators, and Assignment}

\begin{itemize}

  \item What's the difference between
    \mintinline{cpp}{int x;} and \mintinline{cpp}{int x = 0;}?

  \item What are the two functions of the \mintinline{cpp}{<<} operator?  What
    about the \mintinline{cpp}{>>} operator?

  \item What is the value of \mintinline{cpp}{x} in memory after the statement
    \mintinline{cpp}{int x = 3 + 5 * 3 + 10 / 2;} in
    \begin{itemize}
      \item Decimal?
      \item Binary?
      \item Hexadecimal?
    \end{itemize}

  \item If you compile and run the following program

    \inputminted{cpp}{address-of-x.cpp}

    and

    \inputminted{cpp}{address-of-x.cpp.txt}
    
    is printed to the screen (true story), what does this mean?

\end{itemize}


\filbreak
\section*{Program Design}
\begin{itemize}

  \item How many different numbers can you represent using only the fingers of
    one hand?

  \item How would you write a program (just a general design, not actual code,
    for now) to figure this out (without doing any mathematical analysis
    first)?

  \item Is there a simple mathematical formula to find the answer?

  \item Which method do you think would be quicker for the computer to compute?

\end{itemize}


\end{document}

